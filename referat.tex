\documentclass[12pt, a4paper,oneside]{extarticle}
\usepackage[T1,T2A]{fontenc}
\usepackage[utf8]{inputenc}
\usepackage[russian]{babel}
\usepackage{epstopdf}
\usepackage{verbatim}
\usepackage[linesnumbered,boxed]{algorithm2e}
\usepackage[]{graphicx}
\usepackage{setspace}
\usepackage{indentfirst}
\usepackage{hyperref} %url

\usepackage{amsmath} 
 \setcounter{tocdepth}{2} 
\usepackage{floatrow}
% Table float box with bottom caption, box width adjusted to content
\newfloatcommand{capbtabbox}{table}[][\FBwidth]

\onehalfspacing
%\graphicspath{{image/}}
\usepackage[left=2.54cm,right=1.4986cm,
    top=2.0066cm,bottom=2.0066cm,bindingoffset=0cm]{geometry}
\bibliographystyle{unsrt}
\SetKwProg{Op}{Operator}{}{}
\SetKwProg{Func}{Function}{}{}
\SetAlgoNlRelativeSize{0}
\author{Швецов Денис}
\title{Реферат по английскому}
\begin{document}
\thispagestyle{empty}

\begin{titlepage}
\begin{center}
\vspace{-4cm}
%\begin{figure}[h!]
%\center{
%\includegraphics[scale = 1]{msu.png}\\
%}
%\end{figure}

Московский Государственный Университет им. М.В. Ломоносова\\
%\vspace{0.25cm}

Факультет Вычислительной математики и кибернетики\\
%\vspace{0.25cm}

Кафедра автоматизации систем вычислительных комплексов\\
\vspace{5cm}
\vfill

%\Large Швецов Денис Андреевич\\
\vspace{1.25cm}

\textbf{\LARGE Алгоритмы управления перегрузками в центрах обработки 
данных}
\vfill
%ВЫПУСКНАЯ КВАЛИФИКАЦИОННАЯ  РАБОТА
%выпускная квалификационная работа
\end{center}

\vspace{2cm}
\vfill

\begin{flushright}
Швецов Денис\\
\end{flushright}%
\vfill

\begin{center}
Москва, 2017 г.
\end{center}
\end{titlepage}

\section{Введение}

На сегодняшний день центры обработки данных являются важнейшей частью инфраструктуры организаций предоставляющие всевозможные он-лайн сервисы для своих клиентов, например, веб-поиск, торговые площадки, рекомендательные системы. Качество работы этих сервисов напрямую влияет на количество заинтересованных пользователей и следственно на доход. 

Приложения он-лайн сервисов должны работать в реальном времени~--- пользователь вводит запрос в веб-браузере и ожидает незамедлительного отклика (время отклика не должно превышать 300 миллисекунд). 
Так же приложения работают с большими объемами данных (например, весь веб-индекс) для формирования ответа на запрос. Обычно эти данных распределены между тысячами сервером и каждый запрос попадает на все сервера.

Высокие нагрузки на центры обработки данных и жесткие требования приложений порождают жесткие требования на сети в центрах обработки данных. Работа приложений в реальном времени требует низких задержек в сети и устойчивости к разрывам. Также поскольку приложениям надо постоянно обновлять внутренние структуры данных сети должны иметь высокую пропускную способность для длительных потоков. При этом, несмотря на предъявленные требования, в сетях центров обработки данных зачастую используется дешевое сетевое оборудование, например, ToR (Top of the rack) коммутатор, который соединяет между собой сервера в стойке, имеет 48 портом скоростью 1 гигабит в секунду.

В условиях сетей центров обработки данных TCP протокол не достаточно хорошо справляется с поставленными требованиями~\cite{dctcp}, поэтому ведутся работы по разработке протоколов управления перегрузками в центрах обработки данных.
Cуществуют подходы, в которых используются возможности протокола TCP, такие как ECN метка~\cite{dctp, d2tcp}, при этом остальные функции протокола остаются нетронутыми.
Параллельно с этим существуют работы, в которых, разрабатывают протоколы не совместимые с TCP~\cite{d3tcp}, например, Facebook разработала свой протокол управления перегрузками поверх UDP~\cite{facebook}.

\newpage

\section{Центры обработки данных}




%литература

% dctcp
% d3tcp
% d2tcp
% facebook

\end{document}